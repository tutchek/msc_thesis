\newpage
\chap Introduction

\pozn{Zkontrolováno 15.7.}In this thesis, we will study existing approaches for automated logistics. Based on these approaches we will propose an algorithm for a real-life problem of a retail company.

\sec Motivation

Our goal is to develop a system which could be used by a company running a retail store chain with an on-line store that offers thousands of types of goods.

\secc Comparison with a business to business model.

In the business to business model, the customers usually place orders in batches. Each order contains a larger amount of goods. Moreover, because of the batch mode of operation,
one can predict the ordering frequency for each customer. The retail customers usually order a smaller amount of goods; an order with more than 5 items is an exception. 
Furthermore, they place their orders in random periods, which makes it more difficult to predict. The time constraints differ as well. In the business to business model, the time requirements are more strict. As the wholesale customers resell the goods, they need the orders to be completed in a given time limit. Frequently, it is secured with a fee for a late delivery in a contract. In case of retail customers, the situation is seemingly easier. There are no strict limits and the only risk is the possibility of losing
a certain customer (which is not as problematic as the loss of a business partner, since a usual retail customer places about three orders at a retailer
per his life). However, there are other reasons of why the order should be executed as soon as possible. The first reason is that even if the order is executed and the goods
are sent to the customer, they can cancel the order and return the goods. Secondly, the usual standard is that the order is completed the next business day.
The customers expect that and they would prefer a shop which can fulfil these condition. The third reason is that customers expect to find an information about the
stock availability on the website. This closely relates to the second condition. If the information states duration longer than a few days (or if there is a stated ({\em unknown}) delivery time, the customer will place their order in a different shop.

\secc Goal of the thesis

The main goal of a business company is to maximise its profit. We can say that in case the company is focused on selling goods, this goal could be achieved by minimising the purchase prices and maximising the selling prices. However, this is not a complete truth. Firstly, the profit is lowered by all costs which arise during the sale, such as
wages, costs of transportation, running costs of a building where the company is located, electricity and so on. Moreover, the profit can be achieved only if the goods
is sold, and the goods is sold only if it is available and offered to the customer.

This thesis focuses on the logistics of the goods. The goal is to propose an algorithm which deals both with the need to minimise a delay between an order and the goods delivery 
and also with the need to minimise the cost of the transportation. The retail company has usually one or more stores related to the shops where the items are stored.
The whole process is controlled by records kept in the ERP\glos{ERP}{Enterprise resource planning software}\fnote{Enterprise resource planning is a business management software -- usually a suite of integrated applications -- that 
a company can use to store and manage data from every stage of its business.} system. When a customer orders an item, the system generates a shipment order for the respective store.
The storekeeper waits until the requested item is available and then handles the order. Meanwhile, the requested item is relocated from another store or it is ordered from 
a supplier in case it is not available in any of the company stores. The relocation is performed in a similar way. On the source store, there is generated a shipment order, but this time
it instructs to send the goods to the earlier mentioned store.

As we can see, the whole process can be formalised as a set of orders related to the demands arising from the outside world. Still, we must consider, that there are limitations -- 
delivery from one store to another is not instant and can be performed only once in a while. Furthermore, each worker can process only a limited number of items per the working day.
This may lead to several delays in the plan, because an overloaded central hub could break the whole process.

The retail company usually does not operate its own transportation service between the stores. This service is outsourced to delivery companies. Therefore, the company is not -- at least
theoretically -- limited by the number of available vehicles. In fact, the service is conducted by a van that visits the store on a regular basis and collects prepared bags or crates with goods.
We can assume, that a shift of storekeepers cannot exceed the capacity of the van.

Because the delivery companies usually provide a discount on regular routes, not all of the stores are directly connected to each other. When the goods is to be sent to a non-adjacent store,
a proper path through connected stores is selected. The goods itself is packed in a sealed bag which is only moved in an appropriate package to another store on the path. Otherwise, the stores
along the way would have to receive the goods (e.g. generate a warehouse receipt, count all of the items and check them off) and then immediately send the goods again (which means generating a 
picking list, counting and checking off the items). This procedure is time-expensive and could lead to an overload of the store. On the other hand, once a path is planned, it cannot be changed until 
it reaches its destination.

Finally, one aspect shall be mentioned. The shipment orders are sorted by date and are processed sequentially. However, sometimes the queue is jumped. This happens when a salesman wants 
to prefer one particular customer (possibly a valuable customer or a friend of theirs) or simply by mistake. The system should be able to recover from such an error, since as a result, the shipment 
orders could become unresolvable.

\sec Outline of the thesis

In this thesis we propose a system for a real world problem described in the previous section. First, we examine the given problem in chapter \ref[chap:analysis]. The retail company used several 
algorithms, developed by the author of this thesis, to solve the problem. We review them in this chapter as well. Afterwards, we
perform a research survey through logistics systems. In chapter \ref[chap:models] we review several models used in this field and in chapter \ref[chap:strategies] we show solution
strategies for the described models. Next, in chapter \ref[chap:model] we formulate a model suitable for our problem. 
To evaluate the proposed model, we construct a simulator system and use it to compare the model with the currently used algorithm. The benchmarks and their results are described in 
chapter \ref[chap:benchmarking]. Finally, in chapter \ref[chap:conclusion] we discuss the achieved results.

For the better reading experience, in the end of the thesis there is a list of used abbreviation.


