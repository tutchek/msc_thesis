\newpage
\chap Introduction

\pozn{Zkontrolovano: 27.7.2014}In this thesis, we examine a real world problem of a retail company Sunnysoft. The company operates several stores around the Czech Republic together with an electronic shop.
The customers can either order goods in the electronic shop to be delivered by an external carrier company (Czech Post, DPD etc.) or they can reserve the goods in one of the shops.
However, the goods is distributed all around stores. Therefore, the company needs an efficient way to transfer the goods from a store where the goods is available to a store where the goods
is demanded. The company itself does not handle the transfers. Instead, it uses the services of external delivery companies. Therefore, in practical case, the company is not limited by the capacity of transfer 
routes (the delivery company has to only be notified that, exceptionally, it has to use a truck instead of a regular van). However, a store can be overloaded if too many transfers handled in the store are planned. 

Based on the analysis, we propose a formal description of the examined problem, an {\em Automated logistics problem} (ALP) and its variant {\em Automated logistics problem with low-priority demands} (ALP-lp) which 
prevents the overload of the stores\pozn{Pokud zustane ALP-lp2, tak ho sem pripsat...}.

For review, there are several approaches currently used in logistics, which could be used to solve ALP or ALP-lp. In particular, we present approaches based on a Vehicle routing problem, network flows,
constraint programming and mixed integer programming.

To resolve Sunnysoft's problem, we propose three different techniques. In the first one, we relate the problem to a problem of finding a maximal flow through a network and propose an algorithm based
on Ford-Fulkerson algorithm. In the second one, we formulate the problem as a constraint satisfaction problem. Finally, in the third technique, we formulate the problem as a mixed integer programming 
problem. We use state of the art solvers to solve the constraint satisfaction based problem and the mixed integer programming problem.

We perform benchmarking of the proposed algorithms. We perform a qualitative analysis of the solvers and compare the generated plans. We compare the methods to a na\I ve algorihm, as well. We focus on the overall duration of the transfers and on the amount of transfers through the stores. Each demand should be resolved as quickly
as possible, because there are customers waiting for the goods. If the demands are resolved too late, the customers can either cancel the order and order the goods elsewhere or they will be
displeased and will not buy the goods from the company in the future. The amount of transfers through the store should not exceed a particular limit. Otherwise, the store can overload and cease functioning.

Simultaneously, we perform a performance analysis and compare the techniques based on their running time and memory consumption while varying their parameters.

First, we examine the given problem in detail in chapter \ref[chap:analysis]. Afterwards, we formalize an Automated logistics problem and its variant. In chapter \ref[chap:models] we review several
models used in automated logistics and evaluate their ability to solve the problems. Next in chapter \ref[chap:model], we state algorithms which solve the defined problems.
To evaluate the proposed models, we construct a simulation system and perform experiments on the proposed algorithms. The benchmarks and their results are described in 
chapter \ref[chap:benchmarking]. Finally in chapter \ref[chap:conclusion], we discuss the achieved results.

For the better reading experience, in the end of the thesis there is a list of used abbreviation.
