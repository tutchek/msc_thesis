\newpage

\label[chap:analysis]

\chap Problem analysis

\pozn{Zkontrolovano: ne}In this chapter, we will discuss the problem outlined in the previous chapter. Our goal is to describe the process in such a way to be able to formulate a formalization out of it. The analysis
is based on a real situation in the retail store company called Sunnysoft, which uses the ERP\glos{ERP}{Enterprise resource planning software}\fnote{Enterprise resource planning is a business management software -- usually a suite of integrated applications -- that
a company can use to store and manage data from every stage of its business.} software Helios.

\sec Description of a problem

For a retail store company, it is important to have the goods distributed evenly across its stores. This requirement is met easily when a particular item is sold in large amounts. In this case, the item
is usually available in high quantities and can be spread to the company's stores. However, when the item is sold rarely (or the distributor cannot deliver it in desired amounts), its stock supply cannot cover all of the stores sufficiently. In such a case, the goods should be stocked in a store where it gets sold the most. Simultaneously, the store should allow a quick relocation of the goods to another store if needed.

\begfigure
\centerline{\inspic figs/sklady-1.pdf }\nobreak\medskip
\label[fig:stores]\caption/f A situation in the retail store company Sunnysoft. The points represent stores and the lines show possible routes of the goods between the stores. The dashed line represents a route operated by a messenger and normal lines represent a route operated by a regular delivery company.
\endfigure

The stores do not need to be fully interconnected. As we can see in figure \ref[fig:stores], there are no links between Brno and Plzeň. The company uses external delivery company which offers
a special price for an often used delivery route. As there is a lower transfer rate between peripheral branches, it is possible to send the goods through other branches (in this particular case, 
through headquarters in Prague). Moreover, some branches which are close to each other can be used by a messenger travelling between the branches more often (in our case it is between the branches located
in Prague).

\label[secc:analysis:process-driven]
\secc Process driven by documents

Processes related to goods handling are controlled by documents stored in the ERP system. There are several types of these documents. The relevant ones for the problem are the following:
\pozn{shipment order = expediční příkaz, picking list = výdejka, warehouse receipt = příjemka} a {\em shipment order}, a {\em picking list}, and a {\em warehouse 
receipt}. The shipment order is a document ordering the store to deliver an item from a warehouse. The picking list is a document which proves that the item was delivered from the warehouse. Finally,
the warehouse receipt is a document which proves that the item was put to the store. Documents can be related to one another. Typically, there is a following sequence {\em order} $\rightarrow$ {\em shipment order} 
$\rightarrow$ {\em picking list} $\rightarrow$ {\em invoice} in case the order is made by a customer and a different sequence {\em shipment order in store 1} $\rightarrow$ {\em picking list in store 1} $\rightarrow$
{\em warehouse receipt in store 2} in case the goods are transferred from one store another. The picking lists and the warehouse receipts can be either unrealised or realised. The document is in effect 
(and is unchangeable) after its realisation. Such a document cannot be cancelled or
deleted. If there is a need to revert the document, an opposite document has to be created and realised. For each item and each store, the ERP system stores a number representing the amount of pieces in the store. Let us call it {\em InStock}. The {\em InStock} gets updated by the realised picking lists and the realised warehouse receipts. Schematically, it can be expressed as follows:

$$InStock_{g,s} = \sum_{wr \in rWR_s}{\left( \mathbox{\# of items of $g$ on $wr$}  \right)} - \sum_{pl \in rPL_s}{\left(\mathbox{\# of items of $g$ on $pl$}\right)},$$

\noindent where $g$ is a particular commodity, $s$ is a particular store, $rWR_s$ is a set of realised warehouse receipts in the store $s$ and $rPL_s$ is a set of realised picking lists in the store $s$. 
The store operators are responsible for any differences created in the amounts of goods between {\em InStock} and the real amount. Therefore, we may assume that this number is valid and there is no need to 
prepare a backup plan in regards to a situation where there is a missing item.

As we discussed in the previous paragraph, the ERP contains an information about the amount of stocked items in the store. However, this number is not useable in most situations, for the goods can be 
unavailable even when it is physically present in the store. Some of the articles can be reserved for a customer, others could be planned for transfer to another store. Such a reservation
is performed with shipment orders. We define a variable {\em Availability} as follows:

$$Availability_{g,s} = InStock_{g,s} - \sum_{so \in SO_s}{\left( \mathbox{\# of unhandled items of $g$ on $so$} \right)},$$

\noindent where $g$ and $s$ have the same meaning as mentioned above and $SO_s$ is a set of shipment orders in the store $s$. As we can imagine, the availability can be negative, i.e. the demand for an article might be larger 
than the amount in that particular store and therefore, the article needs to be transferred from another store or ordered from the supplier.

This model has one disadvantage, which can lead to the irregularity described in section \ref[sec:analysis:queue-jumping]. Since the ERP system only stores the amount of goods in a store, the amount
of available goods and the list of shipment orders, there is no clear relation between the reserved item and the shipment order. Sometimes, the shipment orders cannot be processed in the order of their 
creation. They could contain another article, which is not available. Therefore, the shipment orders are processed out of their order. Fortunately, there is a way to estimate whether we are able to at least partially
process the shipment order -- we must order the unhandled shipment orders and check, if there is at most $(InStock - 1)$ items on the previous shipment orders.\fnote{In the real application
the ERP has an extension column, where it is possible to find this characteristic -- ``can be processed", ``can be processed partially" and ``cannot be processed" -- precomputed with the database triggers. The operators have 
the ERP preconfigured to show only the shipment orders, which can be at least partially processed.}


\label[secc:analysis:low-priority]
\secc Balancing of goods and low-priority transfers

As we described in section \ref[secc:analysis:process-driven], we can detect the missing goods in order to plan their transfer from another store. However, such a transfer is very often planned with
small amounts of transferred goods. Therefore, we need some heuristics to estimate the need for a particular article in a given store and plan the transfer accordingly to this estimation. However,
these transfers are different from transfers dealing with negative $Availability$ (we will call them ``covering transfers", since they cover the missing items). The covering transfers must be processed as
soon as possible for they indicate that there is a customer waiting for that particular item. The other transfers can be postponed (especially in the case of a store outage described in section 
\ref[secc:analysis:outage]) and will be handled after the covering transfers are processed. Thus, we introduce a label ``low-priority transfer" for a transfer, which does not necessarily need to be processed immediately.

\secc The goods ``on the way"

As described in section \ref[secc:analysis:process-driven], the amount of goods in a particular store can be described using the variables $InStock$ and $Availability$. Unfortunately, this does not
cover all the situations. When an article is sent to another store, it is delivered from a default store but not stocked to its destination store, since there is a realised picking list, but no
corresponding realised warehouse receipt. Such goods is marked for the ERP system as ``in a void". Nevertheless, we can detect such goods by inspecting the unrealised warehouse receipts. If such a receipt is found and has a corresponding realised picking list, it means that there is goods on the way.

It is important to keep track of the goods on the way, for the store can have a negative {\em Availability} but there might be a transfer planned, which could have been already handled by a certain previous action.
Therefore we can introduce a variable $OnTheWay$ with following definition:

$$ OnTheWay_{g,s} = \sum_{wr \in utWR_s}{\left( \mathbox{\# of items of $g$ on $wr$}  \right)},$$

\noindent where the $g$ and $s$ have the same meaning as above and $utWR_s$ is a set of unrealised warehouse receipts in a store $s$ with corresponding realised picking lists in some of the other stores.

It is useful to count with the planned transfer shipment orders. The amount of the goods in such orders is added to the $OnTheWay$ variable. Therefore, we define a $OnTheWay^{*}$ as follows:

$$OnTheWay^{*}_{g,s} = OnTheWay_{g,s} + \sum_{so \in tSO_s}{\mathbox{(\# of unhandled items of $g$ on $so$)}},$$

\noindent where the $g$ and $s$ have the same meaning as above and $tSO_s$ is a set of transfer shipment orders from other stores to store $s$. We exclude the transfer shipment orders for the low priority transfers
described in section \ref[secc:analysis:low-priority]. Since these orders are low-priority, there is no guarantee, that the goods will be dispatched in a reasonable period of time.\fnote{In a real 
situation, we experienced that some stores stopped handling the low-priority transfers at all, since there was an outage of workers. However, the system expected the transfers to be processed so it
did not plan any new transfers (not even the high-priority ones) for the affected goods.} 

With this variable, we can slightly modify the previously defined {\em Availability} and define a variable $Availability^{*}$, which contains the current availability of a particular article in a store
with an increase expected in a few days:

$$Availability^{*}_{g,s} = Availability_{g,s} + OnTheWay^{*}_{g,s}$$

There is no need to transfer the goods to the store, if the $Availability^{*}$ is non-negative. However, we cannot use this variable while planning a transfer to another store, because the goods is 
not present physically.

\label[sec:analysis:pickup]
\secc Pick-up and delivery of items

The goods planned for a transfer are periodically sent to its destination branch. Although, this action cannot be performed continuously as the delivery company stops for a pick-up of the goods only once
in a while. The transfers are therefore performed in batches that are planned on regular bases, which gives us the possibility of creating a schedule out of them. Moreover, for each pair of stores together with a delivery 
type, we have an estimated duration (usually in working days) of the delivery together with the delivery time for the destination store. Ergo, we can compute the shortest path (in time units) for the goods from
all stores to the store where the goods is missing (i.e. $Availability^{*} < 0$).

\label[sec:analysis:irregularities]
\sec Irregularities

The previously described process has some irregularities in the real world, which makes the problem harder to solve. The two most problematic scenarios are  -- queue jumping and storehouse outage.
These irregularities can compromise the whole process and therefore we will pay more attention to them.

\label[sec:analysis:queue-jumping]
\secc Queue jumping

When an item is ordered by a customer or when it is planned for a relocation, it is blocked by a shipment order and is no longer available. However, sometimes a salesman does not respect the blocking and
sells the goods despite its being blocked. The reason for such an action can vary and in fact is not important for our purposes, but the outcome is the same. Either there is a customer whose order was
not processed in time or there is a planned transfer of goods which cannot be processed because the goods is no longer available in that store.

This is a result of a combined system where the inputs are generated either by computers or by humans. Since it cannot be avoided, the system should be able to handle such situations. If there is a planned 
low-priority transfer as described in section \ref[secc:analysis:low-priority], the transfer can be cancelled and the item that was to be transferred can replace the missing goods. If there is no such 
goods available, the system should plan a transfer from another store, where the item is available.

\label[secc:analysis:outage]
\secc Storehouse outage

Stores cannot be always open and operational, since they are tended to by human operators. A state, when the store is not open nor operational or even overloaded, is called a store outage. 
There are several recurring outages -- weekends and state holidays. Sometimes there is an outage caused by an absence of the operator (leave or illness) and sometimes there is another reason for the
outage such as a stock take\pozn{stock take = inventura}. Generally, there are two types of outages -- 1) closed store, 2) overloaded store. 

In the first case, the store cannot receive nor send any goods to another branch. Therefore, it does not make sense to send there any items or plan transfers from that store. We can 
take into an account a planned outage, but with an unplanned outage we have to perform a certain action, for example cancel the planned transfers and plan the transfer from another store.

In the latter case, the store can receive and send goods marked for high-priority transfers (i.e. the transfers dealing with the negative $Availability$ on a store), but will most probably not process the
low-priority transfers.

\sec Automated logistics problem formulation

Based on the description if the previous sections, we now formulate an on-line {\em Automated logistics problem} (ALP\glos{ALP}{Automated logistics problem}), which is to be solved in this thesis. 
Given a 5-tuple $({\bf G}, {\bf S}, {\bf Del}, {\bf D}, OnStock)$, where:

\begitems
* ${\bf S}$ is a set of stores.
* ${\bf G}$ is a set of goods.
* ${\bf Del}$ is a set of delivery routes between stores in ${\bf S}$. A delivery route is a 4-tuple $(s_o, s_d, T_s, T_d)$, where $s_o \in {\bf S}$ is an origin store, $s_d \in {\bf S}$ is a destination
store, $T_s$ is a time of shipment from $s_o$, and $T_d$ is a time of delivery to $s_d$.
* ${\bf D}$ is an ordered set of demands. A demand is a triple $(s, g, amount)$, where $s \in {\bf S}$ is a store where is the demand placed, $g \in {\bf G}$ is a goods demanded by the demand and
an $amount$ is the demanded amount of the goods.
* $OnStock \colon {\bf G} \times {\bf S} \to {\bbchar N}$, a function returning an amount of items of goods, which is present in the store and are not scheduled for transfer.
\enditems

\noindent Find an optimal schedule {\bf Sch}, which is a set of plans. A plan is 5-tuple $(s_o,\penalty0  s_d,\penalty0  g,\penalty0  amount,\penalty0  d)$,  where $s_o, s_d \in {\bf S}$ are an 
origin and a destination store,\penalty0  $g \in {\bf G}$ is a goods, $amount$ is an amount of goods to be transferred and $d \in {\bf D}$ is the demand which is solved by the plan.
The optimality of the schedule {\bf Sch} is measured according to the objective function:

$$\min\quad \sum_{p \in {\bf Sch}}{{\rm duration}(p)}\eqmark$$

\noindent The schedule ${\bf Sch}$ must be compatible with the following constraints:

\begitems \style N
* The sum of planned transfers cannot exceed the amount of goods in the store.

\label[eq:analysis:alp-c1]
$$\forall g \in {\bf G}, \forall s \in {\bf S}\colon \sum_{p \in {\bf Sch}\atop{{\rm goods}(p) = g \atop {\rm origin}(p) = s}}{{\rm amount}(p)} \leq OnStock(g,s) \eqmark$$

* The sum of planned transfers cannot exceed the demanded amount for the particular demand.

\label[eq:analysis:alp-c2]
$$\forall d \in {\bf D}\colon \sum_{p \in {\bf Sch}\atop {\rm demand}(p) = d}{{\rm amount}(p) \leq {\rm amount}(d)}\eqmark$$

\break
* All demands must be processed in their order.
\label[eq:analysis:alp-c3]
$$ \forall p \in {\bf Sch}\colon {\rm demand}(p) = d \Rightarrow \hskip\displaywidth minus 1fill$$
$$\hskip\displaywidth minus 1fill\Rightarrow\forall d' \prec d \wedge {\rm store}(d) = {\rm store}(d')\colon {\rm amount}(d') =\hskip-2em \sum_{\scriptstyle p' \in {\bf Sch}\atop \scriptstyle 
{\rm demand}(p') = d'}{\hskip-2em {\rm amount}(p)} \eqmark$$

* All possible transfers must be scheduled.

\label[eq:analysis:alp-c4]
$$\forall g \in {\bf G}\colon \sum_{\scriptstyle p \in {\bf Sch}\atop\scriptstyle {\rm goods}(p) = g}{{\rm amount}(p)} = \min\left\{\sum_{s \in {\bf S}}{OnStock(g,s)}; 
\sum_{\scriptstyle d\in{\bf D}\atop\scriptstyle {\rm goods}(d) = g}{{\rm amount}(d)} \right\}\eqmark $$


\enditems

\secc ALP with low-priority demands

As discussed in section \ref[secc:analysis:low-priority], we can place low-priority demands. These demands can be ignored, if the store would become overloaded. To model this situation, we
introduce an {\em Automated logistics problem with low-priority demands} (ALP-lp\glos{ALP-lp}{Automated logistics problem with low-priority demands}) as a variation to ALP. We describe only differences
to the original problem.

ALP-lp is a 6-tuple $({\bf G}, {\bf S}, {\bf Del}, {\bf D}, OnStock, Capacity, Priority)$, where

\begitems
* ${\bf G}$, ${\bf S}$, ${\bf Del}$, ${\bf Del}$, ${\bf D}$ and $OnStock$ have the same meaning as in ALP.
* $Capacity \colon {\bf S} \to {\bbchar N}$ is a function which returns a capacity of a given store, i.e. the maximal daily amount of items of goods which can be processed by the store.
* $Priority \colon {\bf D} \to \{high, low\}$ is a function which returns a priority of a given demand.
\enditems

The objective function differs as we need to fulfil two goals, which are in contradiction. The task is to find the shortest schedule, just like in the case of ALP. However, the simple way how to achieve
that state is to ignore all low-priority demands. Therefore ALP-lp is a multiple-criteria optimization problem.

$$\min\quad \sum_{p \in {\bf Sch}}{{\rm duration}(p)}\eqmark$$
$$\max\quad \sum_{p \in {\bf Sch}}{{\rm amount}(p)}\eqmark$$

The generated schedule ${\bf Sch}$ must be compatible with the following constraints:

\begitems \style N
* The constraints \ref[eq:analysis:alp-c1] and \ref[eq:analysis:alp-c2] must be satisfied.
* The order of high-priority orders must be maintained.

$$ \forall p \in {\bf Sch}\colon {\rm demand}(p) = d \Rightarrow \forall d' \prec d \wedge {\rm store}(d) = {\rm store}(d') \wedge Priority(d') = high\colon\hskip\displaywidth minus 1fill$$
$$\hskip\displaywidth minus 1fill {\rm amount}(d') =\hskip-2em \sum_{\scriptstyle p \in {\bf Sch}\atop \scriptstyle 
{\rm demand}(p) = d'}{\hskip-2em {\rm amount}(p)} \eqmark$$

* The capacity of a store cannot be exceeded if possible. Each store cannot process more items of goods per day than its capacity. Otherwise, the store cannot process low-priority demands. We use a 
function $at(p, s)$ which returns an id of a day, when the plan $p$ is processed in a store $s$.

$$ \forall s \in {\bf S}, \forall t \in \{at(p, s) | \forall p \in {\bf Sch}, \forall s \in {\bf S} \}: \hskip\displaywidth minus 1fill $$
$$\hskip\displaywidth minus 1fill\sum_{\scriptstyle p \in {\bf Sch}\atop \scriptstyle at(p, s) = t}{Amount(d)} \geq Capacity(s) \Rightarrow \sum_{\scriptstyle p \in {\bf Sch}\atop 
{\scriptstyle at(p, s) = t \atop \scriptstyle priority(demand(p)) = low}}{Amount(d)} = 0 \eqmark $$

\enditems

\secc ALP-lp with penalties

The ALP-lp can be reformulated to avoid optimization of two contradicting criteria. The {\em Automated logistics problem with low-priority demands and penalties} (ALP-lp2\glos{ALP-lp2}{Automated 
logistics problem with low-priority demands and penalties}) differs from ALP-lp in the objective function. We add a penalty cost for each low-priority demand, which is ignored. The ALP-lp2
is then a 7-tuple $({\bf G}, {\bf S}, {\bf Del}, {\bf D}, OnStock, Capacity, Priority, Penalty)$, where

\begitems
* ${\bf G}$, ${\bf S}$, ${\bf Del}$, ${\bf Del}$, ${\bf D}$, $OnStock$, $Capacity$ and $Priority$ have the same meaning as in ALP-lp.
* $Penalty \colon {\bf D} \to {\bbchar N}$ is a function which returns a penalty if a given low-priority demand is not resolved.
\enditems

The constraints are the same as in ALP-lp
The objective function is stated as follows:

$$\min\quad \left\{ \sum_{p \in {\bf Sch}}{{\rm duration}(p)} + \hskip-2.5em\sum_{\scriptstyle d \in {\bf D} \atop {\scriptstyle {\rm Priority}(d) = low \atop \scriptstyle \not\exists p\in {\bf Sch}: 
{\rm Demand}(p) = d}}{\hskip-2.5em {\rm Penalty}(d)} \right\}\eqmark$$


