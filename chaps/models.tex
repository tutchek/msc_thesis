\newpage

\label[chap:models]

\chap Models

In this chapter, we review useful models and techniques which shall help us formulate a model for the problem described in the chapter \ref[chap:] and approaches commonly used to solve
such models.

\sec Transportation problems

The problem as described in the chapter \ref[chap:motivation] is an instance of transportation problem, as described in \cite[Helmert2003]. This paper describes a Transport task as a 9-tuple
$(V, M, P, l_0, P_G, l_G, fuel_0, cap, road)$, where

\begitems \style N
* $V$ is a finite set of {\em locations},
* $M$ is a finite set of {\em mobiles},
* $P$ is a finite set of {\em portables},
* $l_0:(M \cup P) \to V$ is the {\em initial location} function,
* $P_G \subseteq P$ is the set of {\em goal portables},
* $l_G: P_G \to V$ is the {\em goal location} function,
* $fuel_0:V \to {\bbchar N} \cup \{\infty \}$ is the {\em initial fuel} function,
* $cap: M\to {\bbchar N}$ is the {\em capacity} function, and finally
* $road: M\to {\script P}(V \times V)$ is the {\em roadmap} function.
\enditems

We require that $V, M, {\fam 0 and\ } P$ are disjoint, and that $(V, road(m))$ is an undirected graph for all $m \in M$ (i.e., for all $m \in M$, the relation $road(m)$ is symmetric and irreflexive).

The paper classifies the transportation problems into groups by the value of {\em capacity}, {\em fuel units} and {\em mobiles} parameters as follows:

\bigskip

\hfil\table{|c|c|c|c|}{\crl
\bf capacity & one portable & unbounded & varies \crli
\bf fuel units & one per location & unbounded & varies \crli
\bf mobiles & one mobile & one roadmap & many roadmaps \crl}
\par\nobreak\medskip
\caption/t Values of the parameters of the transportation problems according to which the problems are classified.

\bigskip

we can expect that this problem is NP-complete.

\sec Vehicle routing problems (VRP)

Vehicle routing problem is a transportation problem solving the need to visit a certain amount of destinations with a fleet of trucks. The vehicle routing problem 
is {\it NP-hard} \cite[Lenstra1981]. Still, the vehicle routing problem is widely used in real applications and thus is well researched. Lots of usage examples are 
discussed in \cite[Golden2001].

\sec Capacitated vehicle routing problem (cVRP)

As described in \cite[Dantzig1959], the capacitated vehicle routing problem (originally The Truck Dispatching Problem) may be defined as follows:

\begitems \style N
* Given a set of $n$ ``station points'' $P_i\, (i = 1, 2, \dots, n)$, to which deliveries are made from point $P_0$, called the ``terminal'' 
* A ``Distance Matrix'' $[D] = [d_{ij}]$ is given which specifies the distance $d_{ij} = d_{ji}$ between every pair of points $(i,j = 0, 1, \dots, n)$.
* A ``Delivery Vector'' $(Q) = (q_i)$ is given which specifies the amount $q_i$ to be delivered to every point $P_i\, (i=1,2,\dots,n)$.
* The truck capacity is $C$, where $C > \max{q_i}$.
* If $x_{ij} = x_{ji} = 1$ is interpreted to mean that points $P_i$ and $P_j$ are paired $(i,j = 0,1,\dots,n)$ and if $x_{ij} = x_{ji} = 0$ means that the points are not paired, one
obtains the condition
$$\sum_{j=0}^n{x_{ij}} = 1\quad (i=1,2,\dots,n)$$
since every point $P_i$ is either connected with $P_0$ or at most one other point $P_j$. Furthermore, by definition, $x_{ii} = 0$ for every $i = 0, 1, \dots, n$.
* The problem is to find those values of $x_{ij}$  which make the total distance
$$D = \sum_{i,j=0}^{n}{d_{ij}x_{ij}}$$
a miniumum under the conditions specified in 2) to 5).
\enditems

\sec Vehicle routing problem with time windows (VRPTW)

The VRPTW can be according to \cite[Braysy2005] described as follows:

\begitems \style N
* Let $G=(V,E)$ be a connected digraph consisting of a set of $n+1$ nodes, each of which can be serviced only within a specified time interval or time window and a set of non-negative weights,
$d_{ij}$, and with associated travel times, $t_{ij}$.
* The travel time, $t_{ij}$, includes a service time at node $i$, and a vehicle is permitted to arrive before the opening of the time window.
* Node 0 represents the depot.
* Each node $i$, apart from the depot, imposes a service requirement, $q_i$, that can be a delivery from or a pick-up for the depot.
* The objective is to find the minimum number of tours, $K^\hv$, for a set of identical vehicles such that each node is reached within time window and the accumulated service up to any node
does not exceed a positive number $Q$ (vehicle capacity).
* A secondary objective can be imposed. It is either to minimize the total distance travelled or the duration of the routes.
\enditems

As the cited article states, the VRPTW is widely used and thus well researched problem. Moreover, there exists many heuristics to solve it.

\sec Vehicle routing problem as a job scheduling problem

In \cite[Beck2002] the VRP is reformulated as an scheduling problem. However, these reformulations have not a same performance. As described in \cite[Beck2003], in an optimisation task
the VRP is better in minimising of the total travelled distance whereas the scheduling technology is better in finding the quickest solution. In the existence task was the VRP itself
significantly worse than scheduling technology. Though, the VRP technology can be used to improve results of the scheduling technology.


\sec A classification scheme for vehicle routing ans scheduling problems

As proposed in \cite[Desrochers1990], one can construct a classification scheme which can be used to select a appropriate model to solve a real-life problem. The scheme can be used only
to handle static problems, where the data do not change. The scheme used a formalised language using four or five fields:

\begitems \style N 
* characteristics and constraints relevant only to a single address (the scheme uses the term address rather than customer),
* characteristics relevant to a single vehicle, 
* all program characteristics that cannot be identified with single address or vehicle, 
* one or more objective functions
* description of additional information about specific class of problem instances (this field is optional).
\enditems

For example the above mentioned cVRP expressed in this notation has the following form -- ``$1|m,cap|\,|{\fam0 sum}\,T_i$'' -- which means that there is one depot, $m$ vehicles each with the 
capacity $cap$ and the objective function is sum of route durations.

\sec Daily aircraft routing and scheduling problem (DA\-RSP)

Different approach represents the DARSP described in \cite[Desaulniers1997]. The problem is to generate a schedule for a heterogeneous aircraft fleet covering a set of operational flight legs 
with known departure time windows, durations and profits according to the aircraft type. In the cited paper the problem is defines as follows: {\it ``Given a heterogeneous aircraft fleet, a set
of operational flight legs over a one-day horizon, departure time windows, durations and costs / revenues according to the aircraft type for each flight leg, find a fleet schedule that maximises 
profits and satisfies certain additional constraints.''}

In the paper there are two formulations of the DARSP. The first is based as a Set Partitioning with additional constraints and the second as the time constrained multi-commodity network flow
formulation. For each of the formulations the paper propose a solution strategy. For the Set Partitioning approach the branch-and-bound strategy is feasible and for the multi-commodity network flow
the Dantzing-Wolfe or Lagrangean relaxation embedded in a branch-and-bound search tree.

