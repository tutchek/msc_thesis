\newpage

\label[chap:benchmarking]
\chap Benchmarking

\sec Qualitative analysis

\pozn{Zkontrolovano 15.7.}To evaluate the model proposed in the previous chapter, we compare it with the algorithm described in chapter \ref[chap:analysis]. We perform tests on real data collected during fiscal year 07/2013---06/2014.


\secc Description of the benchmarks

We test the model on data from 1\st November to 24\th December 2013. Due to the increased pre-Christmas demand rate, this period is the most challenging part of the year. We run the 
simulations with the CSP model, described in Chapter \ref[chap:model], Na\"{\i}ve algorithm described in section \ref[secc:analysis:naive] and Ford-Fulkerson based algorithm in section
\ref[secc:analysis:ford-fulkerson].

For each planning algorithm, we run a simulation with a 1-minute, 5-minute and 30-minute-long time step. Each of the configurations is run either with or without the low-priority demands.

Since the CSP model has two parameters, $\alpha$ and $\beta$, we will run simulations with low-priority demands with various values of the parameters.

We measure the following quantities:

\begitems
* Number of resolved demands.
* Total time needed to completely resolve the demands.
* Number of unresolved demands.
* For each store a number of days when the store was overloaded due to exceeding the capacity.
\enditems

We do not compare the algorithms according to the running time, since it is not a critical issue for the problem. However, the CSP solver has a limit in finding the optimal solution. If it does not
find a solution within one minute of the running time, the problem is considered as not being feasible.

\secc Configuration of stores and deliveries

The configuration of stores and delivery routes used in the benchmarks is presented on Figure \ref[fig:benchmarking:deliveries]. There are six stores -- central store (C), e-shop (E), Kovanecká 
street (K), passage Grossmann (G), Brno (B) and 
Plzeň (P). Three of them (central store, e-shop and Kovanecká street) are in one building. Therefore, the delivery is instant and can occur during the whole business day. The Grossmann store is 
located in Prague and its delivery is maintained by a courier company. Therefore, it is not instant but the goods is still delivered in the same day as it is shipped. It is shipped and delivered at a regular
fixed time. Finally, other routes are inter-city routes which occur only once a day at a fixed time. The shipment and delivery times are specific for a particular store.

\begfigure
\centerline{\inspic figs/benchmarking-deliveries-1.pdf }\nobreak\medskip
\label[fig:benchmarking:deliveries]\caption/f The stores and delivery routes used in benchmarking. The row {\em s} represents the conditions for a shipment and the row {\em d} contains the duration
of a transfer along the path and a corresponding delivery time.
\endfigure

\secc Low-priority demands

The low priority demands are generated once a day between 6:00 and 6:59. For each goods $g$ the simulator computes the expected available amount of goods after resolving all of the demands. It uses
the following formula:

$$ExpectedTotalAvailability_g = \sum_{s \in {\bf S}}{InStock_{g,s}} - \sum_{d \in {\bf D}}{Demanded_{g,d}}, \eqmark$$

\noindent where $Demanded_{g,s}$ means the amount of goods $g$ demanded by the demand~$d$. Afterwards, we compute desired amount on the stores. Since we want to balance amounts of goods equally, we use the 
following formula

$$DesiredAmount_g = \left\lceil ExpectedTotalAvailability_g \over |{\bf S}| \right\rceil, \eqmark$$

\noindent and generate the low-priority demand $d$ for each store with 

$$Demanded_{g,d} = DesiredAmount_g.$$

\secc Shortest paths

To compute a set of paths ${\bf P}$ between the stores, we use a Dijkstra's algorithm \cite[Dijkstra1959] on a directed graph displayed on Figure \ref[fig:benchmarking:deliveries]. The paths
are computed for each run once again, since the shortest path depends on a current simulation's date and time. The lengths of paths are computed using the Algorithm \ref[alg:algorithm:delivery-time].

\secc Experimental results

\hfil\table{|l|c|c|c|}{\crl
 & \bf Na\I ve & \bf FF & \bf CSP  \crli
Average waiting time per goods [hour] & & &\crli
Average waiting time per demand [hour] & & & \crli
Total items scheduled & & & \crli
Total demands resolved: & & & \crl
}

\hfil\table{|c|c|c|c|c|}{\crli
$\downarrow\alpha$ / $\beta \to$ & 0 & 1 & 10 & 100 \crli
0 & $\bullet$ & & & \crl
1 & & & & \crl
10 & & & & \crl
100 & & & & \crli}

\sec Performance analysis


\sec Discussion

