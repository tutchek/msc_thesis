\newpage

\label[chap:motivation]

\chap Motivation

Our goal is to develop a system which could be used in by a company which runs a retail store chain with an on-line store, which offers thousands of types of goods.

\sec Comparison with the business to business model.

In the business to business model the customers usually place orders in batches. Each order contains a larger amount of goods. Moreover, because of the batch mode of operation,
one can predict the frequency of ordering for each customer. The retail customers usually order a smaller amount of goods, the order with more than 5 items is exceptional. 
Moreover they place the orders in a random periods, which makes it more difficult to predict them. The time constraints differ as well. In the business to business model
are the time requirements more strict. Since the customers resell the goods, they need the orders to be completed in a given time limit. Frequently it is secured with
late fee in a contract. In the case of retail customers the situation is seemingly easier. There are no strict limits and the only risk is the possibility of losing
of one particular customer (which is not as problematic as losing of the business partner, since usual retail customer places about three orders at the retailer
per his life). However, there are other reasons why the order should be executed as soon as possible. First reason is that even if the order is executed and the goods
are sent to the customer, they can cancel the order and return the goods. Secondly, the usual standard is, such that the order is resolved the next business day.
The customers expect that and they would prefer a shop which can fulfil this condition. The third reason is that customers expect to find an information about the
stock availability on the website. This closely relates to the second condition. If the information states duration longer than a few days (or if there is stated 
{\em unknown}), the customer will place order in a different shop.

\sec Goal of the thesis

The main goal of a business company is to maximise profit. We can say that in a case of the company focused on selling of goods this goal could be achieved by minimising
of purchase prices and maximising of selling prices. However, this is not a complete truth. First, the profit is lowered by all costs which arise during the sale, such as
wages, costs of transportation, running costs of a building where is the company located, electricity and so on. Moreover, the profit can be achieved only if the goods
is sold, and the goods is sold only if it is available and offered to the customer.

The thesis focuses on the logistics of the goods. The goal is to propose an algorithm which deals both with the need to minimise a delay between an order and a delivery 
of the goods and with the need to minimise the cost of the transportation. The retail company has usually one or more stores related to shops which contain goods.
The whole process is controlled by records in the ERP\fnote{Enterprise resource planning is business management software -- usually a suite of integrated applications -- that 
a company can use to store and manage data from every stage of business.} system. When a customer orders an item the system generates a shipment order for the respective store.
The storekeeper waits until the requested item is available and then they handle the order. Meanwhile, the requested item is relocated from another store or it is ordered from 
a supplier if it is not available in all of the company stores. The relocation is performed in a similar way. On the source store there is generated a shipment order, but this time
it instruct to send goods to the earlier mentioned store.

As we can see, the whole process can be formalised as a set of orders related to the demands arising from the outer world. Still, we must consider, that there are limitations -- 
delivery from one store to another is not instant and can be performed only once a while. Furthermore, each worker can process only a limited number of items per working day.
This may lead to several delays in the plan, because an overloaded central hub could broke the whole process.

The retail company usually does not operate its own transportation service between the stores. This service is outsourced to the delivery companies. Therefore, the company is not -- at least
theoretically -- limited by the number of available vehicles. In fact the service is conducted by a van which visits the store on a regular basis and pick-up prepared bags or crates with goods.
We can assume, that a shift of storekeepers cannot exceed the capacity of the van.


