\newpage

\label[chap:implementation]
\chap Implementation

\sec Interaction with the ERP system

The ERP stores its data in a database and the algorithm is able to read all of it. Thus, it has all information as the ERP system does. Moreover, it can interact with the ERP by generating new documents
in the system or by altering or cancelling them.

The data model of the ERP system is not closed, we are able to add new flags and properties to all objects stored in the system. For the purposes of the algorithm, we will define the following flags on
the shipment orders.

\begitems
* {\em Automated logistics} -- a flag indicating that the document was generated by our algorithm.
* {\em Priority} -- a flag indicating whether the transfer of goods is low-priority as described in section \ref[secc:analysis:low-priority].
\enditems

If we would use a different ERP system, which does not allow to alter its data model, we could still store these information separately and link them to ERP's data with their IDs.

\sec CSP solver

We use a ready-made CSP solver to solve a model described in the chapter \ref[chap:model].
There are many CSP solver available. However, the price of the system is important for us. Thus, we prefer an open solution. Based on the CSP solvers survey \cite[Tulacek2009], we chose between a Gecode 
solver \cite[Gecode] and Choco solver \cite[Choco]. Since the Sunnysoft's internal systems are written in PHP and Java programming languages, we will use the Java based CSP solver Choco.

