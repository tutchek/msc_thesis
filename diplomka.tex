\input opmac
\input diplomka_style
% \draft

\citaceAPA

\title={Algorithms for automated logistics}

% \margins/1 a4 (35,40,40,50)mm
\margins/1 a4 (40,25,25,25)mm

\emergencystretch=1em

\fixmnotes\left

% \def\pozn#1{}

\footline={}

\begingroup \fourteenrm
\centerline{\pozn{Gen.: \the\day. \the\month. \the\year}Charles University in Prague}
\medskip
\centerline{Faculty of Mathematics and Physics}
\vfill
\centerline{\sixteenbf MASTER THESIS}
\vfill
\centerline{\picw=64mm \inspic mfflogo.pdf }
\vfill
\vskip5mm
\centerline{\eighteenbf Bc. Michal Tuláček}
\vskip15mm
\centerline{\eighteenbf \the\title}
\vfill
\centerline{Department of Theoretical Computer Science and Mathematical Logic}
\vfill

\setbox0=\vbox{
\+Supervisor of the master thesis: &prof. RNDr. Roman Barták, Ph.D.\cr
\vskip5mm
\+\hfill Study programme: &Computer science\cr
\vskip5mm
\+\hfill Specialization: &Theoretical Computer Science\cr
}
\centerline{\box0}
\vfill
\centerline{Prague 2014}
\endgroup
\eject

\noindent I wish to thank my supervisor, prof. Roman Barták, for his invaluable advice and help. I am grateful for the experimental data provided by the Sunnysoft company, especially, my thanks go to Ing. David
Šilhan and Ing. Martin Strouhal for their support and possibility to test the system in a real environment. My greatest thank-you belongs to you, my sweet Martina. Without your help, support and understanding,
 I would have never finished this thesis.

\vfill

\eject

\null
\vfill

\noindent I declare that I carried out this master thesis independently, and only with the cited 
sources, literature and other professional sources.

\vskip5mm

\noindent I understand that my work relates to the rights and obligations under the Act No.
121\discretionary{/}{/}{/}2000~Sb., the Copyright Act, as amended, in particular the fact that the Charles
University in Prague has the right to conclude a license agreement on the use of this
work as a school work pursuant to subsection 60 (1) of the Copyright Act.

\vskip5mm

\noindent Prague, 31\st July 2014

\vskip15mm

\eject
\parskip=6pt
\bgroup\parindent=0pt
{\chyph
{\bf Název práce:} Algoritmy pro automatizovanou logistiku

{\bf Autor:} Michal Tuláček

{\bf Katedra:} Katedra teoretické informatiky a matematické logiky

{\bf Vedoucí diplomové práce:} prof. RNDr. Roman Barták, Ph.D.

{\bf Abstrakt:} Práce se řeší existující problém optimálního plánování přeskladnění zboží mezi pobočkami maloobchodní společnosti. Cílem je navrhnout systém, který na základě objednávek od zákazníků 
a současných skladových dostupností zboží bude schopen navrhnout optimální plán. Na základě podrobné analýzy problému je v práci formalizován problém automatické logistiky. Po stručném přehledu 
existujících přístupů v oblasti řešení logistických problémů jsou pak navrženy metody řešení založené na programování s omezujícími podmínkami a smíšeného celočíselního programování. Obě metody jsou 
experimentálně navzájem porovnány, a to jak s ohledem na kvalitu nalezeného řešení, tak s ohledem na jejich výkonnost.

{\bf Klíčová slova:} automatická logistika, plánování, rozvrhování
}

\vfil

{\bf Title:} \the\title

{\bf Author:} Michal Tuláček

{\bf Department:} Department of Theoretical Computer Science and Mathematical Logic

{\bf Supervisor:} prof. RNDr. Roman Barták, Ph.D.

{\bf Abstract:} {\em TBA} 

{\bf Keywords:} automated logistics, planning, scheduling
    
\vfil\egroup
\eject

{\chapfont\noindent Contents}
\vskip5mm
\maketoc

\vfill

\eject

\footline={\hss\folio\hss}
\def\zahlavi{\noindent{\it \kern3pt\headrule\kern3pt \the\title}}
\input chaps/introduction-new
\input chaps/problem-analysis
\input chaps/models
% \input chaps/strategies
\input chaps/problem-model
% \input chaps/basic-algorithm
% \input chaps/problem-solution
\input chaps/benchmarking
%\input chaps/discussion
\input chaps/conclusion

\vfill\eject
{\sdef{mt:chap:en}{}\nonum\chap Bibliography\par}
\makebib

\global\chapnum=0
\input appendices/abbrev
\input appendices/content-cd
\input appendices/implementation
\input appendices/simulator
\input appendices/data-files

\bye
