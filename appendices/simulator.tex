\newpage
\label[ap:simulator]
\appendix Description of a simulator

As we described in chapter \ref[chap:analysis], the whole process is controlled by documents. Therefore, based on the data in ERP, we are able to establish the $InStock$ values for each store in any
given moment. Moreover, we have the demands represented by shipment orders and also warehouse receipts which indicate if and when there was an increase in the amount of goods in a given store.

In order to test our algorithm, we created a simulator which generates demands for the planner and applies the generated plan.

\begitems \style N
* {\em Intialization}. For a given date $d$, the simulator sets $Availability_{g,s} = InStock_{g,s}$ based on the history of stock supplies. Then it prepares a sequence of demands ${\bf D}$ and 
a sequence of store replenishments ${\bf R}$, both ordered by date, where the date of the first element of both sequences is after $d$. We introduce a sequence of shedules ${\bf Sch} = \emptyset$.
* {\em Pre-planning phase}. In this phase, the simulator sets $d'=d$ and increases $d$ by a predefined step. Then it constructs a sequence ${\bf D'} \subset {\bf D}$ of demands, which occurred since $d'$.
Next, it updates an $Availability_{g,s}$ with replenishments from ${\bf R}$ which occurred since $d'$ and removes them from ${\bf R}$. Similarly, it processes schedules in ${\bf Sch}$. For each schedule
$sch \in {\bf Sch}$ which is planned between $d'$ and $d$, the simulator lowers the corresponding $OnTheWay_{g,s}$ and increases $Availability_{g,s}$. Then it computes a set of shortest paths ${\bf P}$ 
between the stores, with shipment time $d$.
* {\em Planning}. The simulator runs the selected planner algorithm with ${\bf D'}$ and ${\bf P}$ as an input, producing a set of plans ${\bf Pl}$.
* {\em Update phase}. Each plan $pl \in {\bf Pl}$ contains an information about the origin store $s_o$, the destination store $s_d$, goods to be transferred $s_o$, its $amount$ and a reference to 
a demand $d$, which is resolved by the plan. According to the plan, the simulator decreases the requested amount of goods $g$ in $d$. If the demand $d$ is completely resolved, i.e. it does not demand
any goods, it is removed from ${\bf D}$. Then it decreases $Availability_{g,s_o}$ and increases $OnTheWay_{g,s_d}$ by $amount$. Finally, it adds a new schedule to ${\bf Sch}$ for transfer of $amount$ 
goods $g$ to store $s_d$ with a scheduled time based on the shortest path in ${\bf T}$ between $s_o$ and $s_d$. Afterwards, it continues with phase 2).
\enditems



