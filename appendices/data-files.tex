
\newpage
\appendix Description of data files

We used real data for the benchmarking. The data are generously provided by the company Sunnysoft and contain information about supplies and demands in the fiscal year 2013/14. The data files are located
on the CD provided with this thesis. All files are in the XML format whose format is described bellow. We use a {\tt teletype} script to denote {\tt <tags>} and their {\tt parameters}.

\asec Demands

\begitems
* File name: {\em demands.xml}
* Root element: {\tt <demands>}
* For each demand there is an element {\tt <demand>} with a parameter {\tt date} with a date of placing the demand in the ISO 8601 format \cite[iso8601] and a parameter {\tt store} with the id of a store, 
where the demand is placed.
* For each item of the demand there is an element {\tt <item>} with a parameter {\tt goods} which contains an id of requested goods and a parameter {\tt amount} containing the demanded amount of 
goods.
\enditems

An example of {\em demands.xml}:

\begalgo \tt
\+ <?xml version="1.0"?>\cr
\+ <demands>\cr
\+ \quad&<demand date="2013-07-01T00:02:15+02:00" store="e">\cr
\+ 	&\quad&<item goods="541741" amount="1" />\cr
\+ 		&&<item goods="524796" amount="1" />\cr
\+ 		&&<item goods="527972" amount="1" />\cr
\+ 	&</demand>\cr
\+ 	&<demand date="2013-07-01T00:27:48+02:00" store="e">\cr
\+ 		&&<item goods="538318" amount="1" />\cr
\+ 		&&<item goods="537576" amount="1" />\cr
\+ 	&</demand>\cr
\+&&$\dots$\cr
\+ </demands>\cr
\endalgo


\asec Goods and history of stock availability

\begitems
* File name: {\em goods.xml}
* Root element: {\tt <goods>}
* For each article there is an element {\tt <article>} with a parameter {\tt id}  containing the id of the article.
* For each event in the stock availability history there is an element\break {\tt <history>} with a parameter {\tt date} containing the date of the event in  the ISO 8601 format \cite[iso8601].
* For each store there is an element {\tt <store>} with a parameter {\tt store} containing the id of a store, a parameter {\tt onStock} containing the amount of the article in the store and a 
parameter {\tt onTheWay} containing an amount of the article which is currently transferred from other stores.
\enditems

An example of {\em goods.xml}:

\begalgo  \tt
\+ <?xml version="1.0"?>\cr
\+ <goods>\cr
\+ \quad&<article id="520003">\cr
\+ 		&\quad&<history date="2014-07-01T00:00:00+02:00">\cr
\+ 			&&\quad&<store store="c" onStock="0" onTheWay="0" />\cr
\+ 			&&&<store store="k" onStock="0" onTheWay="0" />\cr
\+ 			&&&<store store="b" onStock="0" onTheWay="0" />\cr
\+ 			&&&<store store="g" onStock="0" onTheWay="0" />\cr
\+ 			&&&<store store="e" onStock="0" onTheWay="0" />\cr
\+ 			&&&<store store="p" onStock="0" onTheWay="0" />\cr
\+ 		&&</history>\cr
\+ 	&</article>\cr
\+ 	&<article id="520848">\cr
\+ 	&&	<history date="2013-07-04T11:48:37+02:00">\cr
\+ 	&&		&<store store="c" onStock="0" onTheWay="0" />\cr
\+ 	&&		&<store store="k" onStock="1" onTheWay="0" />\cr
\+ 	&&		&<store store="b" onStock="3" onTheWay="0" />\cr
\+ 	&&		&<store store="g" onStock="3" onTheWay="0" />\cr
\+ 	&&		&<store store="e" onStock="2" onTheWay="0" />\cr
\+ 	&&		&<store store="p" onStock="2" onTheWay="0" />\cr
\+ 	&&	</history>\cr
\+ 	&&	<history date="2013-07-09T11:51:12+02:00">\cr
\+ 	&&		&<store store="c" onStock="0" onTheWay="0" />\cr
\+ 	&&		&<store store="k" onStock="1" onTheWay="0" />\cr
\+ 	&&		&<store store="b" onStock="3" onTheWay="0" />\cr
\+ 	&&		&<store store="g" onStock="3" onTheWay="0" />\cr
\+ 	&&		&<store store="e" onStock="1" onTheWay="0" />\cr
\+ 	&&		&<store store="p" onStock="2" onTheWay="0" />\cr
\+ 	&&	</history>\cr
\+ 	&&	$\dots$\cr
\+ &</article>\cr
\+ &$\dots$\cr
\+ </goods>\cr
\endalgo

\asec Stores

\begitems
* File name: {\em stores.xml}
* Root element: {\tt <stores>}
* For each store there us an element {\tt <store>} with a parameter {\tt id} containing the identifier of the store used in other files and a parameter {\tt capacity} denoting a maximal daily amount
of goods processed in the store.
\enditems

An example of {\em stores.xml}:

\begalgo\tt
\+ <?xml version="1.0"?>\cr
\+ <stores>\cr
\+ \quad&<store id="c" capacity="50" />\cr
\+      &<store id="k" capacity="50" />\cr
\+      &<store id="b" capacity="50" />\cr
\+	&$\dots$\cr
\+ </stores>\cr
\endalgo

\asec Deliveries

\begitems
* File name: {\em deliveries.xml}
* Root element: {\tt <deliveries>}
* For each delivery there is an element {\tt <delivery>} with mandatory parameters {\tt from} denoting the origin store, {\tt to} denoting the destination store, {\tt time} denoting a time or a time
range of possible shipment, {\tt day} denoting a day of week or a range of days of week of possible shipment (0~=~Sunday, 1~=~Monday and so on), and a {\tt type} which can be {\em instant} if the delivery
does not last any notable time (for example if it is only an accountancy operation\fnote{The stores can be either some physical entity like dedicated room, or it can be only a virtual entity in the
ERP system. In such case, we can move goods from store to store just with generating of respective documents without physical contact with the goods.} or from one room to another) or {\em carrier}, if 
the goods is shipped by an external delivery company. In such case, there are two other parameters. The first is a parameter {\tt duration} with the estimated duration of transfer in days and the 
second is {\tt delivery\_time} with the time of expected delivery to the store.
\enditems

An example of {\em deliveries.xml}:

\begalgo\tt
\+ <?xml version="1.0"?>\cr
\+ <deliveries>\cr
\+ \quad	&<delivery from="c" to="k" time="9:00-17:00" day="1-5"\cr
\+		&\quad&type="instant" />\cr
\+ 	&<delivery from="b" to="e" time="12:00" day="1-5"\cr
\+ 	&&type="carrier" duration="1" delivery\_time="11:00" />\cr
\+ 	&$\dots$\cr
\+ </deliveries>\cr
\endalgo
